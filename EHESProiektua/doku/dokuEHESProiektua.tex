\documentclass{article}
  % M�rgenes
  \textheight    = 20cm
  \textwidth     = 18cm
  \topmargin     = -2cm
  \oddsidemargin = -1cm
  % Sangr�a
  \parindent     =  0mm
  %Paquetes 
  %  deben venir con la distribuci�n TeX 
  %  o se pueden poner en la misma carpeta de este archivo .tex
  \usepackage{amsmath, amssymb, amsfonts, latexsym}
  \usepackage[T1]{fontenc}
  \usepackage[latin1]{inputenc}
  \usepackage{graphicx}
  \title{Dokumentazioa Proiektua}
  \author{Ieltzu,Mikel,Jorge}
  \date{29 de enero de 2010}

  %Inicio del cuerpo del documento
\begin{document}
\newpage
\maketitle
\section{Azaleko Orria}
\newpage
\section{Esparru Teorikoa}
\begin{enumerate}
	\item Dokumentazio eta sintesi lana landutako ikasketa teknikari buruz, alegia:
	\begin{enumerate}[a)]
	\item Support Vector Machines(SVM)
	\item Multilayer Perceptron (MP)
	\item Bayes Network (BayesNet)
	\end{enumerate}
	\item Diseinua laburbildu eta atazen banaketa taldekideen artean eman. Javan garatu den programaren diseinua eta inplementazioari buruzko xehetasun aipagarrienak.
\end{enumerate}
\newpage
\section{Esparru Esperimentala}
\begin{enumerate}
	\item Datuak: datuen deskribapen kualitatibo eta kuantitatiboa 2. ariketako emaitzaksartu.
	\item Aurre-prozesamendua: erabilitako filtroak aukeratzeko motibazioa azaldu eta 3.ariketako emaitzak eman.
	\item Emaitza esperimentalak: ereduek eskaintzen duten kalitatea aztertu. ...
	\item Exekutagarrien exekuzio adibide bat (ikusi 5.1. atala).	
\end{enumerate}
\newpage
\section{Ondorioak eta etorkizunerako lana}
laburbildu lanaren sendotasunak eta ahuleziak.
Aipatu lan honen bitartez atera daitezkeen ondorio nagusiak (bizpahiru). Aipatu nola
hobetu ahal den lan hau berriro hasiko bagina
\newpage
\section{Bibliografia}
Bibliografia edukia indartu edo sostengatzeko emanda dago. Ezinbestekoa
da iturri bibliografikoak aipatzea erabili edo horietara jotzen garen puntuan bertan. Ez ahaztu irudien iturria aipatzen. Testuan zehar esplizituki aipatu ez den iturririk ez sartu Bibliografia atalean. Iturria aipatzean zehaztu ahalbait gehien (liburuko kapitulua edo atala eman), honela beste irakurle batek sakondu ahal izango du iturri horiek kontsultatu.
\newpage
\section{Balorazio Subjektiboa}
(borondatezkoa) eranskin batean atazari buruzko hausnarketa egin. Honako puntuak lagungarriak izan daitezke baina ez dira derrigorrezkoak, beste batzuk ere sar daitezke.
\begin{enumerate}
	\item Atazarekin lortu nahi ziren helburuak lortu dituzue? (ikusi 1. atala)
	\item Batazbestean zenbat denbora eman duzue atazean lanean? Desglosatu: ikasketarako denbora, bilaketa bibliografikoa, softwarearen diseinua eta inplementazioa, txostena.
	\item Talde-lana erabilgarria izan da ataza ebazteko?
	\item Zer sortarazi dizue interesik handiena? Ikasleengan interesa eta motibazioa pizteko iradokizunak eman.   
\end{enumerate}
      
\end{document}
